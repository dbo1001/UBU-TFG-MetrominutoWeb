\capitulo{6}{Trabajos relacionados}

Como se menciona anteriormente en este proyecto, la inicial de Metrominuto proviene de una idea en Pontevedra que tenía como finalidad el fomento de el <<arte de caminar>>~\cite{wiki:metrominuto-pontevedra}.



\subsection{Metrominuto Pontevedra}
FIXME : he encontrado un facebook de metrominuto https://www.facebook.com/metrominuto/  , que no se si puedo hacer mención de algún modo.


\subsection{Aplicaciones similares}
Pasominuto: https://pontevedraviva.com/web/uploads/arquivos/pasominuto.pdf

Conforme iba pasando el tiempo, muchas ciudades copiaron el modelo de Metrominuto que tenía Pontevedra, pero no solo aplicándolo al ámbito urbano, sino que también trasladándolo a otros. Es por esto que de la idea original surgen una serie de aplicaciones o herramientas similares que veremos a continuación.

\subsubsection{Pasominuto}
Esta alternativa a Metrominuto consiste en una guía de recorridos para pasear por <<los espacios más agradables>> de la ciudad, teniendo en cuánta sus distancias, pasos y tiempos. El artículo~\cite{art:pasominuto} menciona hasta un total de 29 recorridos diferentes.
\imagen{pasominuto}{Pasominuto.}

\subsubsection{Metroplayas}
Aplicación considera como <<hijo de metrominuto>> y que sitúa a Pontevedra en el punto equidistante de las mejores playas de las \textit{Rías Baixas}.
\imagen{metroplaya}{Metroplaya.}
