\capitulo{1}{Introducción}
Descripción del contenido del trabajo y del estrucutra de la memoria y del resto de materiales entregados.

El tema principal del proyecto se basa en la mejora para el usuario de los tiempos a la hora de hacer recorridos a traves de las  ciudades. Se basa en la idea de Metrominuto, que consiste en un mapa sinóptico que une diferentes puntos de la ciudad en funcion de la dinstancia existente entre cada uno de ellos. Trata de promover el transito a pie dentro de las ciudades
Este proyecto ofrece la posibilidad de marcar tus propias rutas incluyendo los lugares o destinos que quieras visitar y calcula la mejor forma de recorreerlos en funcion de la distancia para minimizar el tiempo y recorrido empleados en ello.

\subsection{Estructura de la memoria}\label{estructura-memoria}
La memoria se divide en:
\begin{itemize}
\tightlist
\item \textbf{Introducción:} en este apartado se desarrolla de manera breve el tema que se va a tratar en el proyecto, así como la estructura del propio proyecto y los materiales entregados.
\item \textbf{Objetivos del proyecto:} sección en la que se indican los objetivos que se persiguen con la realización del proyecto, tanto técnicos como personales.
\item \textbf{Conceptos teóricos:} apartado en el que se explica todo lo necesario para el correcto entendimiento del tema tratado en el proyecto.
\item \textbf{Técnicas y herramientas:} sección en la que se listan todas las herramientas usadas en el proyecto, así como una breve justificación de su uso en favor de otras herramientas existentes.
\item \textbf{Aspectos relevantes del desarrollo del proyecto:} apartado en el que se explican temas de especial importancia.
\item \textbf{Trabajos relacionados:} sección en la que se indican y se desarrollan brevemente tanto artículos como proyectos que están directamente relacionados con este proyecto.
\item \textbf{Conclusiones y Líneas de trabajo futuras:} apartado en el que se recogen las conclusiones obtenidas una vez finalizado el proyecto, y las posibles mejoras que se pueden hacer en el futuro.
\end{itemize}