\capitulo{3}{Conceptos teóricos}

Aquí se explican los conceptos teóricos en los que se basa el proyecto. Aborda diferentes aspectos, desde los mapas sinópticos de tránsito a la teoría de grafos, así como su dibujado.

\section{Mapa de tránsito}
Un mapa de tránsito consiste en un mapa topológico y esquemático utilizado para mostrar trayectos y estaciones en el ámbito urbano, como puede ser el metro o el autobús. Los elementos principales de este tipo de mapas son:
\begin{itemize}
	\item Líneas de diferentes colores y grosores que indican las distintas líneas del medio de transporte en cuestión.
	\item Iconos o puntos que indican las paradas o estaciones del medio en el que se vaya a viajar.
	\item Diferentes iconología para señalar características significativas.
\end{itemize}
Como aplicación para este proyecto, las estaciones o paradas del mapa corresponderán con los puntos seleccionados por el usuario sobre el mapa, y las líneas serán las distancias correspondientes entre dichos puntos.

\subsection{Harry Beck}
Harry Beck fue un ingeniero electrónico del metro de Londres que trabajaba diseñando diagramas del circuito eléctrico, y que comenzó a diseñar un nuevo mapa de las líneas y estaciones de metro de su ciudad. El objetivo de la solución estaba claro: tenía que ser sencillo de leer para el público y que este pudiese reconocer claramente las distintas estaciones, salidas y traslados.
Realizó varias versiones antes de llegar a la que conocemos hoy en día, como por ejemplo las que podemos observar en las imágenes \ref{fig:MetroLondres1928} y \ref{fig:MetroLondres1933}.
\imagen{MetroLondres1928}{Mapa del metro de Londres de 1928.}
\imagen{MetroLondres1933}{Mapa del metro de Londres de 1933.}

\section{Metrominuto}
El concepto de Metrominuto surgió como resultado de diversas ideas sobre movilidad en la ciudad de Pontevedra. Este concepto hace referencia a un mapa sináptico, como si de un mapa de metro se tratase, que representa las distancias y los tiempos existentes entre los diferentes puntos de una ciudad.
\imagen{MetrominutoPontevedra}{Metrominuto de Pontevedra}

Metrominuto no solo ofrece información de cara a la gente que quiere visitar la ciudad, si no que también fomenta caminar como medio de transporte en una ciudad, donde de una manera sencilla y curiosa nos muestra cómo llegar de un sitio a otro. Caminar, como ya sabemos, es la mejor solución para evitar el gran flujo de automóviles en el área urbana, y lo que ello conlleva: una constante emisión de elementos contaminantes.\\


En los orígenes de este sistema de movilidad se encuentra el estudio, por medio de la técnica DAFO (Debilidades, Amenazas, Fortalezas, Oportunidades):

\begin{description}
	\item[Debilidades:] Como el estado cambiante del tiempo, diferente ritmo al caminar dependiendo de las personas, y la comodidad de coger el coche para moverse.
	\item[Amenazas:] Prejuicios de la población.
	\item[Fortalezas:] Cuidado del medio ambiente, mayor salud y al reducir los desplazamientos en automóvil se produce como resultado una mayor seguridad en los pocos que haya.
	\item[Oportunidades:] Mejorar la ciudad, bienestar.
\end{description}
Estos planos no solo nos incitan a caminar, si no que también incluyen información útil acerca de líneas de autobús, estaciones de ferrocarril o de metro.

\subsection{Proceso de elaboración}
Los pasos a seguir para crear un \textit{Metrominuto} de forma manual son los siguientes:

\begin{steps}
	\item Consiste en la selección, dentro de una ciudad, de los puntos que se quieren representar en el mapa. Estos puntos pueden elegirse en función de su importancia, interés turístico o de los ciudadanos.
	\item Decidir qué ruta peatonal es la mas adecuada para unirlos.
	\item Considerar cómo se va a dibujar el mapa. Puede ser más o menos preciso respecto a la realidad cartográfica.
	\item Situar un punto central que sirva como punto de origen y de orientación para todos los usuarios.
	\item Realizar por medio de herramientas de mapas, como Google Maps en nuestro caso o los mapas de Bing, el cálculo de las distancias entre los diferentes puntos.
	\item Establecer una relación entre las distancias con el tiempo medio que lleva recorrerlas. Tenemos que tener en cuenta que toda la población no camina al mismo ritmo.
	\item Una vez establecidas las diferentes rutas, hacer un estudio sobre ellas para corregir errores que puedan surgir, así como la variación en el tiempo si el terreno no es uniforme o si las condiciones de tráfico y semáforos varía.
	\item Reflejar accidentes naturales o elementos de la ciudad como parques, costa, ríos\dots A través de elementos muy sencillos y con un código de colores al que estamos acostumbrados.
	\item Reflejar aspectos de la movilidad intermodal, es decir elementos como estaciones de metro, autobús, tren, etc. 
	\item Advertir de los espacios con condiciones adversas para personas con problemas de movilidad.
	\item Simplicidad, claridad y facilidad de lectura a la hora de dibujar el mapa.
	\item No sólo mostrar conexiones con el punto central establecido como referencia, sino que también debe aparecer información sobre la interconexión entre los diferentes puntos.
\end{steps}

El objetivo de este proyecto es automatizar este proceso recogido en el documento publicado por el Concello de Pontevedra~\cite{metrominuto}, ya que actualmente los Metrominutos existentes se realizan de esta forma totalmente manual. Con el proceso automatizado sería el mismo usuario quien realice su propio Metrominuto con los puntos de interés personalizados que él decida. Evitando que aparezcan puntos o información que no le resulta interesante.


\section{Teoría de grafos}
Es aquella rama de las matemáticas que, junto con la ciencia de computación se encarga de estudiar las propiedades de los grafos. Primeramente debemos saber que un grafo \textit{G=(V,E)} es un conjunto de vértices o nodos unidos por enlaces llamados arcos.
Existen varios tipos de grafos, pero en ese proyecto se han usado grafos no dirigidos para realizar las distintas operaciones con los nodos, es decir, los nodos corresponden con los puntos o lugares que selecciona el usuario y los arcos hacen referencia a la distancia existente entre ellos~\cite{wiki:tgrafos}. 

\subsection{Uso de grafos}
El resultado final de este proyecto es un mapa sinóptico formado por lugares y el <<camino>> o distancia entre ellos.\\


Dicho mapa se genera tras realizar varias operaciones (listadas a continuación) con grafos, donde los nodos serán los lugares previamente seleccionados y los arcos los trayectos que unen dichos lugares.

\begin{steps}
	\item Calcular las distancias de todos con todos.
	\item Eliminar un nodo de la lista.
	\item Minimum spanning tree~\cite{wiki:minimum-spanning-tree} del grafo obtenido en el paso 2.
	\item Sumar un voto a cada arco resultante.
	\item Repetir pasos 2, 3 y 4 hasta que se haya eliminado cada nodo del grafo una única vez.
\end{steps}

Este proceso se detalla en la sección de aspectos relevantes del proyecto (ver \nameref{sub:grafos-aspectos-relevantes}).


\section{Scalable Vector Graphics (SVG)}
SVG o gráficos vectoriales escalables, es un término que hace referencia a un formato de gráficos vectoriales bidimensionales bien sean estáticos o dinámicos en formato XML~\cite{wiki:xml}.
Permite tres tipos de elementos:
\begin{enumerate}
	\item Elementos geométricos vectoriales, como líneas, rectas o círculos.
	\item Imágenes de mapa de bits/digitales.
	\item Texto.
\end{enumerate}

Las ventajas que presenta SVG son que a estos elementos se les pueden aplicar diferentes estilos, agrupar o transformar bien sea antes de la compilación o dinámicamente. 


Este formato de representación gráfica será el elegido para el dibujado de los mapas sinópticos, ya que con librerías como \nameref{sub:svgwrite} y \nameref{sub:snap} podemos manipular los distintos elementos.


