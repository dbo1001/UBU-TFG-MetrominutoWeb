\apendice{Especificación de diseño}

\section{Introducción}
En este apartado de la documentación se expone el diseño que ha dado lugar a la aplicación, el cual incluye el diseño de las distintas estructuras de datos, el diseño procedimental y diseño arquitectónico.

\section{Diseño de datos}
La estructura del proyecto podemos dividirla en varias partes ya que, debido al uso de bibliotecas como la de Google Maps o de la biblioteca de Networkx para la generación de grafos, se han usado todos o parte de los datos que nos devuelven como resultado de las consultas.

\subsection{NetworkX}
En este proyecto se usa networkX para generar, modificar y visualizar grafos, en los cuales el elemento <<\textit{nodes}>> representa los diferentes puntos seleccionados por el usuario, y el elemento <<\textit{edges}>> representa las conexiones entre ellos.

\subsection{Google API}
Google se ha usado para la obtención de todos los datos necesarios para el cálculo de distancias y tiempos, así como para la selección de los diferentes puntos en el mapa. Para ello, Google proporciona un API para \textit{Python} y otro para \textit{JavaScript}. Las funciones que se han usado han sido:
\begin{itemize}
	\item \texttt{distance\_matrix(orígenes, destinos)}: devuelve las distancias de cada origen con cada destino.
	\item \texttt{directions()}: devuelve las direcciones que hay que seguir para llegar de un punto a otro.
\end{itemize}


\section{Diseño procedimental}
En esta sección se explican las interacciones más destacadas de la aplicación. La principal de ellas es, dentro de la visualización de los puntos, cuando se pasa a la vista del mapa sinóptico, ya que aquí entran en juego todas las partes de la aplicación.
// Diagrama de secuencia.


Cuando el usuario selecciona y guarda una serie de puntos sobre el mapa, estos pasan al servidor de tal manera que, con el API proporcionado por Google se evalúan las distancias de unos nodos a otros y se genera un grafo. A continuación, se realizan diferentes operaciones con dicho grafo para evaluar los diferentes caminos más cortos para recorrer todos los nodos. \\
Como resultado final, obtenemos un mapa sinóptico con la representación de estos puntos y de los caminos existentes entre ellos.


\section{Diseño arquitectónico}

La estructura del proyecto esta condicionada por el tipo de proyecto que es. Se trata de una aplicación web y por ello se ha seguido el patrón MVC (Modelo - Vista - Controlador), el cual permite separar en 3 componentes diferentes los datos, el interfaz y la lógica de la aplicación. 
\imagen{mvc}{Diagrama Modelo-Vista-Controlador}

\section{Maquetación}
Al tratarse de una aplicación web es muy importante que la apariencia de la misma este cuidada y sea adaptable a cualquier tipo de dispositivo.
\\

Para la construcción de las distintas páginas de este proyecto, se ha heredado una estructura común definida en el fichero \textit{base.html}. Este fichero importa diferentes librerías, como \textit{Bootstrap} y \textit{Vue}. También añade la barra de navegación y otros elementos comunes a todas las páginas. Es por ello que desde el comienzo del proyecto y con la finalidad de evitar futuros problemas a la hora de ordenar y alinear los elementos \textit{HTML} conviene tenerlo bien estructurado y ordenado. 
\\
A la hora de situar los diferentes elementos HTML en la página, como por ejemplo los botones de \textit{<<Aceptar>>} y \textit{<<Cancelar>>}, es importante tener en cuenta la usabilidad de los mismos.

