\apendice{Especificación de diseño}

\section{Introducción}
En este apartado de la documentación se expone el diseño que ha dado lugar a la aplicación, el cual incluye el diseño de las distintas estructuras de datos, el diseño procedimental y diseño arquitectónico.

\section{Diseño de datos}
La estructura del proyecto podemos dividirla en varias partes ya que, debido al uso de bibliotecas como la de Google Maps o de la biblioteca de Networkx para la generación de grafos, se han usado todos o parte de los datos que nos devuelven como resultado de las consultas.

\subsection{NetworkX}
En este proyecto se usa networkX para generar, modificar y visualizar grafos, en los cuales el elemento <<\textit{nodes}>> representa los diferentes puntos seleccionados por el usuario, y el elemento <<\textit{edges}>> representa las conexiones entre ellos.

\subsection{Google API}
Google se ha usado para la obtención de todos los datos necesarios para el cálculo de distancias y tiempos, así como para la selección de los diferentes puntos en el mapa. Para ello, Google proporciona un API para \textit{Python} y otro para \textit{JavaScript}. Las funciones que se han usado han sido:
\begin{itemize}
	\item distance\_matrix(orígenes, destinos): devuelve las distancias de cada origen con cada destino.
	\item directions(): devuelve las direcciones que hay que seguir para llegar de un punto a otro.
\end{itemize}


\section{Diseño procedimental}


\section{Diseño arquitectónico}

La estructura del proyecto esta condicionada por el tipo de proyecto que es. Se trata de una aplicación web y por ello se ha seguido el patrón MVC (Modelo - Vista - Controlador), el cual permite separar en 3 componentes diferentes los datos, el interfaz y la lógica de la aplicación. 
\imagen{mvc}{Diagrama Modelo-Vista-Controlador}
