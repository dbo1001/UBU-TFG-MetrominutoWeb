\apendice{Especificación de Requisitos}

\section{Introducción}
En este apéndice se explican y especifican tanto los requisitos funcionales como los no funcionales del proyecto, así como los objetivos del proyecto.

\section{Objetivos generales}

\begin{itemize}
	\item Crear una aplicación cliente -- servidor que permita la creación automática de Metrominutos.
    \item Control de usuarios.
    \item Permitir a los usuarios control sobre el mapa, de manera que puedan mover o eliminar los puntos seleccionados.
    \item Ofrecer al usuario un mapa final claro y sencillo.
    \item Que la aplicación final sea útil para el fomento de esta actividad.
\end{itemize}

\section{Catálogo de requisitos}

\subsection{Requisitos funcionales}
\begin{itemize}
	\item \textbf{RF-1 Control de usuarios}: la aplicación debe permitir el control de usuarios.
	\begin{itemize}
		\item \textbf{RF-1.1 Firebase Auth:} La aplicación debe poder hacer uso de los servicios de autenticación de Firebase.
	\end{itemize}
	\item \textbf{RF-2 Generación de mapa:} La aplicación debe ofrecer la selección de distintos puntos sobre el mapa para generar un mapa personalizado.
	\begin{itemize}
		\item \textbf{RF-2.1 Visualización de puntos sobre Google Maps:} La aplicación debe poder hacer uso de las operaciones, así como acceder a los servicios de \textit{Google Maps} proporcionados por el API.
		\item \textbf{RF-2.2 Selección de puntos:} Seleccionar sobre el mapa distintos puntos, hasta un máximo de 15.
		\item \textbf{RF-2.3 Selección como centrales:} Una vez creado un marcador, el usuario debe poder darle una mayor importancia a este punto en el recorrido.
		\item \textbf{RF-2.4 Eliminación de un punto cualquiera:} El usuario debe poder eliminar los puntos seleccionados.
	\end{itemize}
	\item \textbf{RF-3 Creación de SVG:} Generación de un mapa sinóptico a través de los puntos seleccionados en el \textit{RF-~2}, con formato SVG, que permite la interacción con dicho mapa.
	\begin{itemize}
		\item \textbf{RF-3.1 Añadir y eliminar arcos:} El usuario debe poder seleccionar el mapa o grafo que más se adecue a su necesidad, con más o menos recorridos (arcos).
		\item \textbf{RF-3.2 Mover puntos y etiquetas:} Posibilidad de que el usuario recoloque tanto los puntos como las etiquetas de los mismos en el mapa.
		\item \textbf{RF-3.3 Cambiar el nombre de los puntos:} El usuario debe poder cambiar las etiquetas referentes al nombre de los puntos.
		\item \textbf{RF-3.4 Exportación y descarga:} El usuario debe poder descargar el mapa que ha generado.
	\end{itemize}
\end{itemize}


\subsection{Requisitos no funcionales}
\begin{itemize}
	\item \textbf{RNF-1 Usabilidad:} La aplicación tiene que poder usarse de forma sencilla y debe ser intuitiva.
	\item \textbf{RNF-2 Compatibilidad:} La aplicación tiene que poder ser compatible con los diferentes navegadores.
	\item \textbf{RNF-3 Responsividad:} La aplicación debe poder adaptarse al tamaño de la pantalla.
	\item \textbf{RNF-4 Facilidad en el despliegue:} La aplicación debe poder ser desplegada con facilidad en un servidor.
	\item \textbf{RFN-5 Mantenibilidad:} Debe ser sencillo añadir nuevas funcionalidades.
\end{itemize}

\section{Especificación de requisitos}
En esta sección, se presentan los casos de uso de la aplicación.
\imagen{Diagrama de caso de uso - sitema}{Diagrama de casos de uso: General.}
\imagen{Diagrama de caso de uso - RF2}{Diagrama de casos de uso: RF2 -- Generación de mapa.}
\imagen{Diagrama de caso de uso - RF3}{Diagrama de casos de uso: RF3 -- Creación de SVG.}

