\capitulo{1}{Introducción}
-- Movilidad Urbana.



-- Metrominutos, lo que son y dónde se aplican




-- Lo que va a hacer el proyecto: ayudar a hacer metrominutos automáticos


El tema principal del proyecto se basa en la mejora de la movilidad de los peatones en las ciudades a la hora de transitar por ellas a pie. Se centra en la idea de Metrominuto~\cite{metrominuto}, que consiste en un mapa sinóptico que une diferentes puntos de la ciudad en función de la distancia existente entre cada uno de ellos. Su propósito es animar a los ciudadanos a moverse por la ciudad, lo cual supone beneficios en muchos aspectos: tanto de salud, como de contaminaciones.

Actualmente ya existen ciudades con Metrominutos como Pontevedra (pionera en esta idea), Sevilla, Madrid o León, pero este proyecto lo que trata es de automatizar este proceso de creación de mapas de manera que es el propio usuario quien selecciona los puntos que van a aparecer en él.

//Hablar sobre la movilidad urbana y sobre el fomento del turismo y la movilidad a pie ?? Quizás algún gráfico?


