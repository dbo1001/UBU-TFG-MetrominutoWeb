\capitulo{5}{Aspectos relevantes del desarrollo del proyecto}

En este apartado se van a recoger y explicar los aspectos más importantes del desarrollo del proyecto. Desde las implicaciones de las decisiones que se tomaron, hasta los numerosos y variados problemas a los que hubo que enfrentarse.

\subsection{Elección del proyecto}

El año pasado fui uno de los privilegiados de poder disfrutar de una beca Erasmus, en concreto con destino en la ciudad polaca de Gliwice~\cite{wiki:gliwice}.
Esto me llevó a conocer nuevas culturas pero también a conocer nuevas ciudades, y en la mayoría de ocasiones el tiempo del que disponíamos para recorrerlas era muy breve. Por ello, al ver las posibilidades que ofrecía el resultado final de este TFG me llamó la atención, ya que es una aplicación que de haberla tenido nos habría ahorrado muchos desplazamientos quizá inútiles al recorrer estas ciudades sin una ruta fija.

\subsection{Formación}

Para poder realizar el proyecto se necesitaban unos conocimientos no adquiridos sobre desarrollo web, tanto de la parte de servidor en Flask como la parte del cliente en HTML, CSS y JavaScript. Además de para aprendizaje, los recursos se han usado también como material de consulta durante el desarrollo.

Para la parte del servidor se siguieron los libros y tutoriales:
\begin{itemize}
	\item Flask Web Development~\cite{grinberg2014flask}.
	\item The Flask Mega-Tutorial (2017)~\cite{grinberg-mega}.
\end{itemize}

Para la parte del cliente se utilizaron principalmente los siguientes
materiales:
\begin{itemize}
	\item W3Schools Tutorials~\cite{w3schools}.
\end{itemize}

A medida que se añadían nuevas herramientas al proyecto, su documentación
oficial también ha sido consultada en varias ocasiones, están disponibles en:
\begin{itemize}
	\item Documentación de Flask~\cite{doc:flask}.
	\item Documentación de Bootstrap~\cite{doc:bootstrap}.
	\item Documentación de NetworkX \cite{SciPyProceedings_11}.
\end{itemize}
