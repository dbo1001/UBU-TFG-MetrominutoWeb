\apendice{Documentación de usuario}

\section{Introducción}
En este apéndice se explica los requisitos que debe cumplir el usuario para ejecutar la aplicación, como lanzarla y como usarla.

\section{Requisitos de usuarios}
Al tratarse de una aplicación web los requisitos que debe cumplir el usuario son los siguiente:
\begin{itemize}
	\item Navegador web instalado.
	\item JavaScript activo en el navegador.
	\item \textit{Cookies} activas en el navegador.
\end{itemize}

La aplicación esta diseñada para que sea usable tanto en ordenadores como en dispositivos móviles, debido a que su funcionalidad esta orientada a que los principales usuarios estén en continuo movimiento.

\section{Instalación}
Debido a que se proporciona una aplicación web no es necesario instalarla para poder usarla. Se puede acceder a través de \url{http://tfgmetrominuto.azurewebsites.net/}

\section{Manual del usuario}
Incluir algo parecido a la página de ayuda de la aplicación.
Incluir comentarios sobre la interacción al mover los puntos.

Para el uso de esta aplicación es necesario haber iniciado sesión. Para ello, hacer click sobre el botón \textbf{Iniciar sesión} en la parte superior derecha de la página de inicio, y elegir el método de autenticación preferido.


A continuación, aparece una página con un mapa en el que se pueden seleccionar, clickando sobre el mismo, distintos puntos de interés hasta un máximo de quince. Una vez seleccionados dichos puntos, existe la posibilidad de borrar uno o varios si los hemos seleccionado por error, o de marcarlos como centrales. Esto significa que estos puntos marcados van a tener más peso en el cálculo de los distintos recorridos. A continuación, pulsamos sobre el botón \textbf{Generar Mapa}.


Se mostrará una nueva página con una representación de los puntos que hemos seleccionado y el tiempo que hay entre ellos, indicado por las líneas que los unen. El color de cada una de estas depende directamente del tiempo que representan. Por ejemplo, los trayectos inferiores a 5 minutos se mostrarás de color verde, mientras que los superiores a 15 minutos serán de color rojo. Están disponibles varias posibilidades en función del número de puntos que hayamos incluido en el mapa, y para cambiar de una otra es necesario desplazar el punto de la parte superior izquierda sobre la barra en la que está, como vemos en la siguiente figura:
\imagen{slider}{Número de arcos.}


Una vez seleccionado el mapa más adecuado, pulsar sobre el botón \textbf{Enviar}. Esta acción mostrará una nueva pantalla con el mismo mapa que hemos seleccionado previamente, pero con la posibilidad de interaccionar con él:
\begin{enumerate}
	\item Cambiar los puntos de posición: al mover un punto, todas las etiquetas relacionadas con este punto y sus conexiones se recalcularán.
	\item Cambiar las etiquetas de texto de posición.
	\item Cambiar el nombre de alguno de los puntos.
\end{enumerate}

Se recomienda seguir este orden a la hora de realizar cambios sobre el mapa, ya que si mueven primeramente las etiquetas se pierde la posibilidad de que cuando se mueve el punto se recalculen.


Por último, si se pulsa sobre el botón \textbf{Exportar} se descargará el mapa que se está visualizando.

