\apendice{Especificación de Requisitos}

\section{Introducción}
En este apéndice se explican y especifican tanto los requisitos funcionales como los no funcionales del proyecto, así como los objetivos del proyecto.

\section{Objetivos generales}

\begin{itemize}
	\item Crear una aplicación cliente -- servidor que permita la creación automática de metrominutos.
    \item Ofrecer control de usuarios (posible idea futura junto con lo comentado en la reunión de las API KEYs?)
    \item Permitir a los usuarios control sobre el mapa, de manera que puedan mover o eliminar los puntos seleccionados.
    \item Ofrecer al usuario un mapa final claro y sencillo.
    \item Que la aplicación final sea útil para el fomento de esta actividad.
\end{itemize}

\section{Catálogo de requisitos}

\subsection{Requisitos funcionales}
\begin{itemize}
	\item \textbf{RF-1 Control de usuarios}: la aplicación debe permitir el control de usuarios.
	\begin{itemize}
		\item \textbf{RF-1.1 Integración con Google Auth:} La aplicación debe poder hacer uso de cuentas de google para el control de usuarios.
	\end{itemize}
	\item \textbf{RF-2 Control sobre el mapa:} La aplicación debe poder ofrecer la selección de distintos puntos sobre el mapa.
	\begin{itemize}
		\item \textbf{RF-2.1 Integración con el API de Google:} La aplicación debe poder hacer uso de las operaciones, así como acceder a los servicios de \textit{Google Maps} proporcionados por el API.
		\item \textbf{RF-2.2 Creación:} Debe poder añadir tantos puntos como desee.
		\item \textbf{RF-2.3 Modificación:} Una vez creado un marcador, futuro nodo del grafo, el usuario debe poder moverlo.
		\item \textbf{RF-2.4 Eliminación:} El usuario debe poder eliminar los puntos.
	\end{itemize}
	\item \textbf{RF-3 Generación de mapas sinópticos:} Generación de un mapa sinóptico a través de los puntos seleccionados en el \textit{Requisito Funcional~2}.
	\begin{itemize}
		\item \textbf{RF-3.1 Modificación:} El usuario debe poder mover los puntos del mapa.
		\item \textbf{RF-3.2 Apariencia:} Posibilidad de cambio de color de los puntos, así como de las líneas que los unen.
		\item \textbf{RF-3.3 Eliminación:} El usuario debe poder eliminar dichos puntos.
		\item \textbf{RF-3.4 Elementos adicionales:} Dibujado de elementos adicionales como pueden ser ríos, parques, o estaciones públicas.
	\end{itemize}
\end{itemize}


\subsection{Requisitos no funcionales}
\begin{itemize}
	\item \textbf{RNF-1 Usabilidad:} La aplicación tiene que poder usarse de forma sencilla y debe ser intuitiva.
	\item \textbf{RNF-2 Compatibilidad:} La aplicación tiene que poder ser compatible con los diferentes navegadores.
	\item \textbf{RNF-3 Responsividad:} La aplicación debe poder adaptarse al tamaño de la pantalla.
	\item \textbf{RNF-4 Facilidad en el despliegue:} La aplicación debe poder ser desplegada con facilidad en un servidor.
	\item \textbf{RFN-5 Mantenibilidad:} Debe ser sencillo añadir nuevas funcionalidades.
\end{itemize}

\section{Especificación de requisitos}


