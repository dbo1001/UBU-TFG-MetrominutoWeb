\capitulo{6}{Trabajos relacionados}

Como se menciona anteriormente en este proyecto, la inicial de Metrominuto proviene de una idea en Pontevedra que tenía como finalidad el fomento de el <<arte de caminar>>~\cite{wiki:metrominuto-pontevedra}.



\subsection{Artículos relacionados}

\subsubsection{Al menos 57 ciudades han copiado el Metrominuto pontevedrés~\cite{art:metromin-ciudades}}
Este artículo explica que no sólo ciudades españolas han replicado la idea del Concello de Pontevedra, si no que dicha idea se ha extendido a ciudades europeas como Florencia y Cagliari  en Italia o Poznan en Polonia, como podemos ver en la figura~\ref{fig:poznan}.
\imagen{poznan}{Metrominnuto de Poznan}

FIXME: tiro por artículos?


\subsection{Aplicaciones similares}
Pasominuto: https://pontevedraviva.com/web/uploads/arquivos/pasominuto.pdf

Conforme iba pasando el tiempo, muchas ciudades copiaron el modelo de Metrominuto que tenía Pontevedra, pero no solo aplicandolo al ámbito urbano, si no que también trasladándolo a otros. Es por esto que de la idea original surgen una serie de aplicaciones o herramientas similares que veremos a continuación.

\subsubsection{Pasominuto}
Esta alternativa a Metrominuto consiste en una guía de recorridos para pasear por <<los espacios más agradables>> de la ciudad, teniendo en cuanta sus distancias, pasos y tiempos. El artículo~\cite{art:pasominuto} menciona hasta un total de 29 recorridos diferentes.
\imagen{pasominuto}{Pasominuto.}

\subsubsection{Metropalyas}
Apllicación considera como <<hijo de metrominuto>> y que situa a Pontevedra en el punto equidistante de las mejores playas de las \textit{Rías Baixas}.
\imagen{metroplaya}{Metroplaya.}



FIXME : he encontrado un facebook de metrominuto https://www.facebook.com/metrominuto/  , que no se si puedo hacer mención de algún modo.