\capitulo{1}{Introducción}

Hoy en día la movilidad en los centros urbanos es mayoritariamente motorizada y ocupa alrededor del 70\% del espacio público. Es por ello, que en las ciudades o áreas urbanas, a excepción de algunas áreas residenciales, las zonas peatonales escasean, impidiendo, al menos dificultando, los desplazamientos a pie. Como solución a este problema, surge en la localidad de Pontevedra la idea de Metrominutoo~\cite{metrominuto}.
\\


Un Metrominuto consiste en un mapa sinóptico que une diferentes puntos de la ciudad en función de la distancia existente entre cada uno de ellos, y cuyo propósito fundamental es justamente promover que la gente camine y se desplace a pie. Nuevamente, como mencionaba arriba, para que esto pueda hacerse realidad el propio diseño de las áreas urbanas influye notablemente.


Actualmente ya existen ciudades con Metrominutos como Pontevedra (pionera en esta idea), Sevilla, Madrid o León, pero este proyecto lo que trata es de automatizar este proceso de creación de mapas de manera que es el propio usuario quien selecciona los puntos que van a aparecer en él.
\\


Este proyecto se centra en el desarrollo de una aplicación web para la generación automática de estos mapas sinópticos, o mejor dicho, Metrominutos. De este modo, el propio usuario será el encargado de decidir qué puntos aparecen en el mapa y personalizarlo a su antojo: podrá mover puntos, cambiar los nombres si fuese necesario o elegir qué trayectos aparecen o no. Así, el usuario podrá generar en cualquier momento un mapa que se adapte a su necesidad real, eliminando elementos que no son de su interés o que, simplemente, no quiere visitar.