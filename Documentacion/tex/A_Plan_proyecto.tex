\apendice{Plan de Proyecto Software}

\section{Introducción}
En esta fase se analiza en detalle todo aquello necesario para que un proyecto se desarrolle de una forma eficaz y lo mas llevadera posible de manera que surja el menor numero de imprevistos posible.
Es en esta fase donde se estima el tiempo que va a llevar, el trabajo y el dinero requerido para que el proyecto se lleve acabo.

\section{Planificación temporal}
Esta parte del proyecto es aquella en la que se planifica cómo va a ir avanzando el proyecto en función del trabajo requerido para cada una de las tareas de las que consta el mismo.
Concretamente se estima el tiempo, es decir, cuáles van a ser los plazos para desarrollar determinadas tareas.
Para esta planificación o estimación se han empleado los conceptos generales de la metodología ágil Scrum, ya que en este caso en el proyecto sólo hay un único desarrollador a parte de los tutores. Las lineas generales que se han aplicado de esta metodología de gestión han sido:
\begin{itemize}
	\tightlist
	\item
	Desarrollo marcado por sucesivos \emph{sprints} marcados por dos reuniones: una al principio de cada uno y otra al final. Normalmente en la reunión de finalización de un \emph{sprint} se marcaban los objetivos y la planificación del siguiente (siendo la primera reunión).
	\item
	La duración de los \emph{sprints} fue de dos semanas.
	\item
	Cada \emph{sprints} pruce un resultado o incremente del proyecto final.
	\item
	En cada \emph{sprint} se dividía el objetivo final en distintas tareas mas pequeñas.
	\item
	Las tareas se planificaban y estimaban en un tablero.
\end{itemize}

\subsection{Sprint 0. ()}
En esta reunión el objetivo fundamental fue la presentación, a grandes rasgos, de en que iba a consistir el proyecto por parte de los dos tutores: Alvar Arnaiz Gonzalez y Cesar Ignacio Garcia Osorio.
No se definió ninguna tarea, ya que simplemente se trataba de acordar si se había entendido bien el objetivo del proyecto.

\subsection{Sprint 1. (29/10/2019 - 12/11/2019)}
Esta reunión fue la primera en la que se comenzó a hablar de los requisitos del proyecto y de sus detalles para la planificación.
Los objetivos de este sprint fueron: crear correctamente el repositorio en GitHub, elegir el entorno de desarrollo que se iba a utilizar y su posterior configuración para ejecutar una aplicación web con Flask, hacer un primer proyecto (HolaMundo) como primera toma de contacto con este framework, y por ultimo, que nuestro proyecto final incorporase ya un mapa proporcionado por el API de Google en el que fuésemos capaces de seleccionar diferentes puntos (marcadores).


\subsection{Sprint 2. (13/11/2019 - 26/11/2019)}
Los objetivos principales de este Sprint fueron: guardar e imprimir (tanto en el cliente como en el servidor) los distintos marcadores seleccionados en el mapa, dibujar la ruta entre los distintos puntos seleccionados, seguir los estándares de programación, empezar a completar la documentación del proyecto, y cambiar de Visual Studio Code a Pycharm (licencia profesional).
Ademas de estos objetivos, debido a la posibilidad de añadir al proyecto nuevas funcionalidades y metodologías de Docker, se decidio cambiar de Sistema Operativo a Linux.
Durante este sprint se siguió un curso o tutorial sobre Flask de Miguel Grinberg, en el cual, a medida que iba avanzando encontré varias mejoras para aplicar a mi proyecto y que fui incorporando.
Al final de este sprint, dos de los objetivos no se consiguió por completo, ya que daban algunos errores y se opto por una funcionalidad menor: en vez de dibujar la ruta entre todos los puntos seleccionados, sólo se dibujaba entre el primero y el ultimo; y al pasar los distintos marcadores, mediante un POST al servidor no podia pasar un objeto. Estas dos funcionalidades quedaron pendientes para el siguiente Sprint.

\subsection{Sprint 3. (27/11/2019 - 10/12/2019)}
El principal objetivo de este sprint fue poner al dia la documentación al mismo tiempo que se seguía el tutorial de Flask mencionado en el sprint 2.
Se encontraron distintas mejoras para realizar, asi como la posibilidad de añadir el fichero requirements.txt, que después serviría para instalar las diferentes librerías utilizadas en el proyecto.
También se acabaron las tareas que quedaron pendientes el sprint anterior.

\subsection{Sprint 4}
Los objetivos de este sprint fueron: mejorar el envío de los datos al servidor, buscar documentación y evaluar los resultados de las distintas funciones que proporciona el API de Google para Python, y ser capaces de diferenciar los datos "útiles" de dichas funciones.


\subsection{Sprint 5}


\section{Estudio de viabilidad}

\subsection{Viabilidad económica}

\subsection{Viabilidad legal}


