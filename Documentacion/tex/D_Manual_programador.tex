\lstset{basicstyle=\ttfamily,
	showstringspaces=false,
	commentstyle=\color{red},
	keywordstyle=\color{blue},
	frame=single
}

\apendice{Documentación técnica de programación}

\section{Introducción}
En este apéndice se va a definir todo aquello que es necesario conocer para que se pueda continuar con el desarrollo del proyecto, desde su estructura hasta una breve descripción de como instalar la aplicación y configurar nuestro entorno de trabajo para llevar a cabo el desarrollo.

\section{Estructura de directorios}
La estructura del proyecto se divide en:
\dirtree{%
	.1 / Directorio raíz.
	.2 Documentacion/ -Documentación del proyecto.
	.3 img/ -Imágenes de la documentación.
	.3 tex/ -Secciones de la documentación.
	.3 anexos.pdf -Anexos del proyecto.
	.3 memoria.pdf -Memoria del proyecto.
	.2 HolaMundo/ -App web básica.
	.2 Metrominuto/ -Aplicación web.
	.3 static/ -ficheros JavaScript.
	.3 templates/ -ficheros HTML.
}

\section{Manual del programador}
En este apartado se explican los puntos a tener en cuenta por futuros desarrolladores que tengan la intención de mantener o mejorar el proyecto.

\section{Aplicaciones utilizadas}
Para el desarrollo de este proyecto se tuvieron en cuenta principalmente dos editores de texto y dos herramientas para mantener el control de versiones.

\begin{itemize}
	\item \textbf{Visual Studio Code} 
	\item \textbf{PyCharm}
	\item \textbf{GitHub}
	\item \textbf{GitCraken}
\end{itemize}
Después de analizar y configurar ambos editores de texto, se llegó a la conclusión de que era mucho mas cómodo y útil utilizar PyCharm, ya que ofrece una configuración mas sencilla, además de permitir importar diversas librerías de una manera mas amigable. También ofrece la posibilidad de seguir los estándares de programación \textit{PEP8} y \textit{ECMAScript}.

\section{Instalación y configuración}
Para la instalación del proyecto se explicarán los pasos a seguir en un sistema operativo de Linux, que en este caso se trata de la versión Linux Mint 18.3 Sylvia.

\subsection{Python}

Este proyecto está desarrollado con la versión 3.6.3. Python se puede descargar desde el siguiente enlace: 
\url{https://www.python.org/downloads/}

\subsection{Instalación y configuración de PyCharm}
Este \textit{IDE} tiene distribución para Linux, además de permitir a estudiantes usar la versión profesional. Para su instalación podemos usar la \cite{snap store} de Linux, que en caso de no tenerla instalada tenemos que ejecutar el siguiente comando:

\begin{lstlisting}[language=bash,caption={Instalar snapd}]
$ sudo apt update
$ sudo apt install snapd
\end{lstlisting}
Después de tener instalado esto, ejecutaríamos:
\begin{lstlisting}[language=bash,caption={Instalar PyCharm}]
$ sudo snap install 
	pycharm-community|professional --classic
\end{lstlisting}

Una vez instalado \textit{PyCharm}, la forma más cómoda de obtener el código del proyecto es mediante \textit{git}, usando para ello el comando:
\begin{lstlisting}[language=bash,caption={Descargar el repositorio}]
$ git clone <url_del_repositorio>
\end{lstlisting}
Siendo \url{https://github.com/gpm0009/TFG_MetrominutoWeb.git} la URL del 
repositorio.

Para instalar las dependencias ejecutar el comando:
\begin{lstlisting}[language=bash,caption={Instalar requirements.txt}]
$ pip install -r requirementes.txt
\end{lstlisting}

\subsection{Claves de Google}
Para la obtención de un \textit{Google API KEY} es necesario obtener los credenciales en  \url{https://console.developers.google.com/apis/credentials}.
Después, dichos credenciales deben activarse para las APIs que utiliza este proyecto:
\begin{itemize}
	\item Directions.
	\item Distance Matrix.
	\item Geocoding.
	\item Maps JavaScript.
	\item Places.
\end{itemize}

Una vez que la tenemos, debemos incluirla en el proyecto. Para ello:

\begin{lstlisting}[language=python,caption={Añadir API\_KEY}]
google_maps=googlemaps.Client(key='GOOGLE_API_KEY')
\end{lstlisting}

Además, no hay que olvidar incluirla en los templates:
\begin{lstlisting}[language=html,caption={Añadir API\_KEY a los templates}]
<script
 src="https://maps.googleapis.com/maps/api/
	 js?key=API_KEY&libraries=places" 
	 type="text/javascript"</script>
\end{lstlisting}

También podemos incluirla como variable de entorno en nuestro editor. De esta manera nos aseguramos de no compartirla al realizar los commints en el control de versiones.
\\
En este caso, en PyCharm se configura de la siguiente manera:
\begin{enumerate}
\item Abrir selector Run Configuration (arriba a la derecha)
\item Edit Configurations...
\item Environmental variables
\item Add or change variables, then click OK 
\end{enumerate}
  
\subsection{\TeX studio}
Esta herramienta para la compilación de documentación \LaTeX{} permite la instalación de diccionarios para aplicar las reglas al texto. Para ello, debemos acceder a:

\begin{lstlisting}[language=bash,caption={Añadir diccionario}]
Options -> Configure TeXstudio
Language checking
\end{lstlisting}

Una vez ahí, vemos que nos ofrece dos opciones para buscar diccionarios: \url{https://extensions.openoffice.org/de/search?f[0]=field_project_tags} o \url{https://extensions.libreoffice.org/extensions?getCategories=Dictionary&getCompatibility=any}. Elegimos cualquiera de ellas y descargamos el paquete del diccionario que queramos, y después lo importamos.


\section{Librerías}

\subsection{NetworkX}
Como ya he mencionado, esta biblioteca nos permite trabajar de una forma muy amplia y completa con grafos. A lo largo del proyecto se han usado:
\begin{itemize}
	\item \texttt{graph()}: para crear un grafo no dirigido al que se añadirán nodos y arcos.
	\item add\_node(): Diferentes nodos junto con atributos como la posición obtenida del API de Google y el nombre.
	\item add\_edge(): Arco que conecta dos nodos. Además, los arcos contienen atributos como la distancia real que hay de nodo a nodo o el número de votos que tendrá.
	\item get\_edge\_attributes(): para obtener los valores de un atributo perteneciente a los arcos. Devuelve una lista que contiene el nodo de origen, el nodo destino y el atributo deseado.
	\item edges(data\=True): devuelve los arcos junto con los atributos.
	\item nodes(data\=True): devuelve los nodos junto con los atributos.
	\item minimum\_spanning\_edges(): devuelve un iterador con los arcos que forman el grafo de tal manera que la suma de distancias es la mínima.
	\item draw\_networkx(): para dibujar el grafo.
\end{itemize}





\section{Compilación, instalación y ejecución del proyecto}


\section{Pruebas del sistema}
